\documentclass[final,3p,times,authoryear]{elsarticle}

 \makeatletter
 \def\ps@pprintTitle{%
  \let\@oddhead\@empty
  \let\@evenhead\@empty
  \def\@oddfoot{}%
  \let\@evenfoot\@oddfoot}
 \makeatother
 \begin{document}     
 \title{Metro Tunnel - Applied Learning Opportunities research proposal}
 \author{Kerry~A.~Nice
 \\Transport, Health, and Urban Design Hub, Faculty of Architecture, Building, and Planning, University of Melbourne, Victoria 3010, Australia}
 \date{\today}
 \maketitle

\section{Background}\label{sec:introduction}
Urban heat is a challenge to policy makers, urban planners, and residents of urban areas. Heat waves have been shown to cause significant increases in mortality and morbidity as daily temperatures cross thresholds \citep{Nicholls2008}. Concurrently, projections of future climate change suggest an increase in the frequency, duration, and intensities of heat waves \citep{Alexander2009}. In addition, the design of urban areas exacerbates the impacts of heat waves and urban heat through urban heat island (UHI) effects, in which increased storage of daytime heat in urban areas is amplified by impervious surfaces, an increased amount of anthropogenic heat, and the reduction of shade providing and moisture providing vegetation canopies \citep{Coutts2012}.

Other impacts of urban design on human health can be seen in journey mode choices. A growing body of evidence confirms an association between the quality of the built environment and levels of physical activity, with amenities such as shading have been shown to increase the walkability of an area \citep{Millington2009,Gallin2001,LSA2003}. Increased active journeys lead to lower body mass index (BMI) \citep{Davison2008,Lee2008b}, reduced risk of cardiovascular disease, stroke and hypertension \citep{WHO2010,Warburton2006} and decreased health care costs \citep{Stephenson2000}. Addressing these problems requires both a means to understand the factors causing the problems as well as a method to test possible redesign strategies for the best possible health outcomes.

\section{Project overview}
Human thermal comfort provides a link between both of these pressing health issues, the impacts of heat waves and poor thermal comfort discouraging active transport modes.  This project fits within two of the applied learning initiatives, inner urban cooling and making Melbourne multi-modal. We propose to develop a design decision support system (DDSS) integrating thermal comfort and multi-modal transportation based on agent based modelling (ABM). ABM allows interactions between intelligent autonomous agents to be modelled within a defined space such as a rail station catchment. Agents navigate the street and public transport network interacting with the urban environment (including heat mapped urban canyons and urban features such as vegetation canopies). By changing the make-up of the modelled area, the parameters of the simulation, and varying interactions of the agents, different urban design scenarios can be simulated and tested.

To develop the modelling areas, a wide availability of urban datasets including GIS datasets and street view imagery and new processing techniques can be used to extract the urban morphology to build modelling domains. These combined with urban climate thermal comfort modelling output allows large urban areas to be heat mapped and incorporated into modelling scenarios. Further modifications to these urban areas allows thermal comfort impacts of urban design changes (i.e. increases to the vegetation canopy) to be quantified and added to the ABM DDSS as a interactive layer. Validation of this thermal comfort modelling can be facilitated through observations from University Square, as this area has been monitored during during the construction phase.

ABM modelling can then be used to quantify the impacts of these urban redesigns on heat health and thermal stress (such as during heat waves) and the impact on multi-modal transport choices (including active transport modes) in the catchment. Issues such as bikes on trains and its impact on the enabling cycling as a primary transport mode on longer journeys can be modelled. 

An important consideration that is often overlooked during long term construction projects is the impacts the project can have on the urban areas during the construction. The DDSS AGM modelling platform can incorporate stages of the construction project and examine ways to minimize the impacts to both thermal comfort and active transport mode choices during the construction phases and after completion of the project. 


\section*{References}\label{sec:ref}
  \bibliographystyle{elsarticle-harv} 
  \bibliography{library}


\end{document}
